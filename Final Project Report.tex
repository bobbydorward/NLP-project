\documentclass[12pt]{article}
\usepackage{amssymb,amsfonts,amsthm,mathtools,float,color}
\usepackage{caption,subcaption,tikz,gensymb,pgfplots}\usepackage{setspace}
\doublespacing
\pgfplotsset{compat = newest}
\DeclarePairedDelimiter\ceil{\lceil}{\rceil}
\DeclarePairedDelimiter\floor{\lfloor}{\rfloor}
\usepackage[hmargin=1in,vmargin=1in]{geometry}
\newcommand{\si}{\sigma}
\DeclareMathOperator{\lb}{lb}
\DeclareMathOperator{\ls}{ls}
\DeclareMathOperator{\rb}{rb}
\DeclareMathOperator{\rs}{rs}
\DeclareMathOperator{\LB}{LB}
\DeclareMathOperator{\Max}{Max}
\DeclareMathOperator{\LS}{LS}
\DeclareMathOperator{\RB}{RB}
\DeclareMathOperator{\RS}{RS}
\DeclareMathOperator{\hook}{hook}
      \theoremstyle{plain}
      \newtheorem{theorem}{Theorem}[section]
      \newtheorem{lemma}[theorem]{Lemma}
      \newtheorem{corollary}[theorem]{Corollary}
      \newtheorem{proposition}[theorem]{Proposition}
      \theoremstyle{definition}
      \newtheorem{definition}[theorem]{Definition}
      \theoremstyle{remark}
      \newtheorem{remark}[theorem]{Remark}
      \theoremstyle{plain}
      \newtheorem{conjecture}{Conjecture}[section]
\def\vs{\vspace{1\baselineskip}}
\def\svs{\vspace{.5\baselineskip}}
\def\bs{\vspace{5\baselineskip}}
\def\zik{\mathcal{Z}_{i,k}}
\def\setzik{\{\mathcal{Z}\}_k}
\def\Z{\mathcal{Z}}
%\newcommand{\tre}[1]{\textcolor{red}{}}
\newcommand{\tre}{\textcolor{red}}
\newcommand{\tbl}[1]{\textcolor{blue}{}}
\newcommand{\tgr}[1]{\textcolor{green}{}}
\DeclareMathOperator{\WLB}{WLB}
\DeclareMathOperator{\WLS}{WLS}
\DeclareMathOperator{\WRB}{WRB}
\DeclareMathOperator{\WRS}{WRS}
\DeclareMathOperator{\WF}{WF}
\DeclareMathOperator{\W}{\mathcal{W}}
\def\zik{\mathcal{Z}_{i,k}}
\def\setzik{\{\mathcal{Z}\}_k}
\def\Z{\mathcal{Z}}
\def\M{{\bf M}}
\def\l{\ell}
\def\multiset#1#2{\ensuremath{\left(\kern-.3em\left(\genfrac{}{}{0pt}{}{#1}{#2}\right)\kern-.3em\right)}}
\newcommand{\Rtil}{\widetilde{R}}
\newcommand{\LBtil}{\widetilde{\LB}}
\newcommand{\LStil}{\widetilde{\LS}}
\newcommand{\RBtil}{\widetilde{\RB}}
\newcommand{\RStil}{\widetilde{\RS}}
\newcommand{\gauss}[2]{\genfrac{[}{]}{0pt}{}{#1}{#2}} 
\newcommand{\TODO}{\tre{TODO!!!!!!}}
\newcommand{\interesting}{\tre{SUPER INTERESTING\\}}
\newcommand{\figureout}{\tre{Need to figure this out\\}}
\newcommand{\N}{\mathbb{N}}
\newcommand{\norm}[1]{\left\lVert #1 \right\rVert}


\begin{document}
\author{Bobby Dorward and Ryan Wilson}
\title{Semi-supervised Document Summary}
\maketitle


\section{Introduction}
	In our final project, we looked at the problem of single-document summarization. That is, we attempted to write a program to read in a document, and output a short summary of the document. Our summaries are general summaries rather than query-based summaries, and we generate extracts rather than abstracts for our summaries (meaning that our summaries are made using specific informative sentences extracted from our document, rather than being fully self-generated using the extracted raw information).
	
	Since summarization is a complex task, it poses a particularly interesting natural language processing problem. For a human, summarization requires a moderately complex understanding of a document; critical information must be understood and prioritized, combined into a cohesive set of concepts, then put into words in a way that will both maintain the meaning of the high-priority concepts, and make sense in terms of the rules of the language in which the summary is written. These steps are modified, of course, by probabilistic approaches that simplify the nature of each of these problems.
	
	We consulted the Jurafsky \& Martin textbook, specifically Chapter 23, entitled \emph{Question Answering and Summarization}, as well as the article \emph{A Statistical Approach for Document Summarization} from the Department of Computer Engineering at the Fr. C. Rodrigues Institute of Technology. Chapter 23 of the textbook discusses multi- and single-document summarization and gives a brief discussion of different ways to solve different subproblems of the summarization problem. The article mentioned above attempts to accomplish its own specific task (query-based abstract summarization based on a given query, generated using two or more web documents, which are selected based on the provided query). We consider their techniques and alternative modified approaches to the problem formulated by ourselves throughout the course of the project, as well as their results, to learn more about the nature of these algorithms and how to further refine our own approach.
\section{Implementation}
Document summarization can be broken down into 3 subproblems; content selection (extracting summary-worthy material), information ordering (putting together the selected content), and sentence realization ("cleaning up" the ordered information such that it is concise, cohesive, and sensical). For content selection, we start by assigning each sentence in the document an "information content" score. Then, we rank these sentences by information content, and finally select the top X\% of ranked sentences, where X is a parameter set in our code (10\% by default).

The first step is content selection, selecting sentences from the document to extract. To achieve this goal, we used a semi-supervised approach recommend by Jurafsky and Martin. There are two main ideas to this approach. The first is that we can think of sentences as bags of words, vectors in $n$-dimensional space, where $n$ is the number of unique words in our document $d$. The document is then a cluster and a sentence is worthy of being extracted if it is near the center of the cluster. In order for this approach to work, we need a measure of distance between sentences, which brings us to the other main idea of the method. We want to weight certain words in our sentence as more important than others based on two criteria. The first is that words that appear very frequently in the document are probably very important to the document. This is the \emph{term-frequency} measure
$$tf(w) := \text{number of occurrences of word $w$ in $d$}.$$ However, lots of words appear frequently without contributing any content. Some examples are articles, prepositions and other structural words. Therefore, we also weight words by whats called the \emph{inverse document frequency}
$$idf(w) := \log\left(\frac{N}{N_w}\right),$$
where $N$ is the total numbers of documents in our corpora and $N_w$ is the number of documents in the corpora that contain $w$. A nice feature of the $idf$ is that it deals with stop words. If a word is in every document (e.g., ``the") then 
$$idf(the) = \log(1) = 0$$
and the word is not included in our estimates.
We use these functions to create a feature vector $v_s = (v_1,v_2,\dots,v_n)$ for each sentence $s$, where
$$v_i =\begin{cases}
tf(w_i)idf(w_i) & \text{ if } w_i\in s\\
0 &\text{otherwise.}
\end{cases}$$
We then can define our distance, which is really a measure of similarity because we want to choose sentences with a large distance from other sentences. The distance Jurafsky and Martin recommend is the cosine distance
$$tf-idf-cosine(s_1,s_2) := \frac{ v_{s_1} \cdot v_{s_2}}{\norm{v_{s_1}}\norm{v_{s_2}}}.$$
With our distance function we can define a measure of how central a sentence is to our cluster, which is just the average distance to all other sentences in the document
$$centrality(s) := \frac{1}{|d|}\sum_{s'\in d} tf-idf-cosine(s,s'),$$
where $|d|$ is the number of sentences in $d$. Thus, a sentence is good for our summary if $centrality(s)$ is large relative to the other sentences in the document. Therefore, we compute the centrality of each sentence and choose the best $X$\% of sentences in $d$, where we currently have the parameter $X=10$, though this could be changed depending on the desired summary length. These best sentences form the extract summary.


	Next, we implemented information ordering. Standard practice for this is very simple- most often, sentences from such a summary are simply ordered in the same relative order they had in the original document. However, one of our goals that developed as we worked on the project was to experiment with slightly different approaches. We did one such experiment with information ordering. We attempted to use an $n$-gram approach to look for higher probability sentence transitions, i.e., we attempted to slightly rearrange sentence ordering to reflect sentences to accommodate any sentence transitions which, according to our $n$-gram approach, are likely to appear. 
	
The intuition behind this approach was that some sentence endings and beginnings have meaning that directly relates to the progression of ideas; for example, the word ``however" is often used in the beginning of a sentence to indicate a counterexample or contrasting statement to the sentence before. Therefore, if we have a \emph{sentence transition $n$-gram}, meaning an $n$-gram that includes words from two different sentences, that is ``... death penalty. However," and we have two sentences in our summary: ``There was a great deal of support for the death penalty." and ``However, a new study indicates that the death penalty is less humane than previously anticipated", we might want to put those two together. Since there is a logical progression from the first sentence to the second, and this is reflected by the structure of the n-gram transition, we hope that making small adjustments to the ordering of the sentences based on these $n$-gram transitions would yield a more sensical and logical progression of ideas. 

Unfortunately, this method proved unsuccessful. Data on sentence transition $n$-grams was too sparse to glean any meaningful information on the relationships we were looking for. Additionally, this approach seemed incompatible with single-document summarization; when adjustments were made, they were more confounding than helpful. This might be a good starting point for implementation of multiple-document summarization, when we want to do significant re-ordering, but for our purposes we found original sentence order to be more effective.

Finally, we have the problem of sentence realization. For this problem, we attempted to create a new technique using only minimal information on the algorithms commonly used to solve the problem. We knew that parse trees were usually used to accomplish this, short of the use of complex semantic analysis that is beyond the scope of this project. So, using this information, we set out to find our own way to prune sentences. 


We accomplish this by first parsing the sentence and then trying to prune subtrees of the parsetree. The advantage of pruning subtrees over just random chunks of the sentence is that the grammatical structure of the sentence should be less effected by pruning subtrees. In an ideal world, the algorithm would prune unnecessary clauses out of the sentence, while retaining the core meaning.

To achieve this, we leveraged the code we had already written to try to prune sentences. For each sentence, the algorithm finds all the different ways to prune the parse tree of that sentence. It then does an in order traversal of all these trees to extract the pruned sentence from each tree. With pruned sentences, we can use the centrality measure we used originally for content selection to find the best pruned sentence.

The results of this first attempt at pruning can be seen in Figure~\ref{badsummary}. As we can see, the pruning is too aggressive, pruning sentences down to their keywords, but creating gibberish in the process. We therefore needed to refine this process.

The next and final idea was to weight sentences which had been pruned less higher. The reasoning being that we should really only prune a lot of the sentence if the information to be pruned is really worthless. Pruning less aggressively also leads to sentences which are more likely to retain their grammatical structure.

We can see in Figure~\ref{goodsummary} that this final implementation strikes a pretty good balance between pruning and retaining the meaning of each sentence.



\section{Data}
	The corpus that we use for the training of all parts of our algorithm (and to find documents for summarization) in the final version is the Open American National Corpus (a.k.a. the OANC; we use files in Graph Annotation Format). This corpus consists of about 9,500 articles on a variety of different subjects, of lengths usually on the order of a few dozen paragraphs. 
	
Initially, we intended to use a variety of different genres of document and train only on documents of the same category when summarizing a given document. However, as we developed our own approach to the problem, namely our modification of the sentence ordering subproblem, we decided it would be more pertinent to use all of the documents at our disposal to train on. For starters, we concluded that the logical progression of ideas that are hopefully represented by sentence transition $n$-grams would be the same regardless of genre. Specifically, we felt that the structural meaning of words like "however" or "additionally" would be the same in any English document. However, as we will discuss when we analyze our results, sparseness of data for sentence transition n-grams also played a part in our decision to stick to training on all of our data at once. 

	Prior to our final implementation, we also tried using the Reuters Corpus, which consists of nearly all articles ever published on reuters.com. However, this corpus was much smaller than the OANC, and as a result was thrown out. We also considered using the Penn Treebank, prior to discovering the cost, and we also considered using Project Gutenberg. However, Project Gutenberg does not have a single centralized corpus that is available to the public, and we also decided that novels would not be the most useful tests of our algorithm, given their length. So, we finally settled on the OANC, and it has proven successful for our purposes.
\section{Results}

	In Figure~\ref{goodsummary} we have an example of a summary produced by our final implementation running on a document about the history of Budapest.

\begin{figure}
	``Around the third century b.c., Hungary was occupied by a Celtic-Illyrian tribe known as Eraviscans, refugees from wars in Greece.
Hungary remained beyond the reach of western Europe until the first century a.d., when the Roman empire's legions advanced and pushed its northeast frontier to the Danube.
Hungary was effectively dismembered: the north and west fell to the Habsburgs; Transylvania became a so-called independent principality under Turkish auspices; and central Hungary came under direct Turkish rule.
In 1867, under a compromise designed to curtail home-rule agitation, the Austro-Hungarian empire was established and Hungary was finally granted its own government, though key ministries were shared with the Austrians.
Meanwhile, the aftermath of war, as dictated by the 1920 Treaty of Trianon, was to cost Hungary very dearly.
Hitler's Germany, meanwhile, provided investment in Hungarian industry and a market for Hungarian farm produce, and earned a grudging admiration from the Hungarians for its defiance of the World War I allies.
In 1940 Hungary allowed the German army to cross its territory and, as a reward, they temporarily recovered parts of its former lands from Romania and Yugoslavia.
Hungary was the first country to draw back its Iron Curtain, dismantling the barbed wire along its Austrian border and allowing East Germans to escape to the West.
In 1997 Hungary was offered membership in the NATO."
\caption{The unpruned summary for a document on the history of Budapest.}\label{rawsummary}
\end{figure}
	
\begin{figure}``Around the third century b.c., Hungary was Greece. Hungary remained beyond the reach of western Europe until the first century a.d., when the Roman legions advanced and pushed its northeast frontier to the Danube. Hungary was effectively dismembered: the north and west fell to the Habsburgs; Transylvania became a so-called independent principality under Turkish auspices; and central Hungary came under rule. In 1867, under a compromise designed to curtail home-rule agitation, the empire was established and Hungary was finally granted its own government, though key ministries were shared with the Austrians. Meanwhile, the aftermath of war, as dictated by Trianon, was to cost Hungary very dearly. meanwhile, provided investment in Hungarian industry and a market for Hungarian farm produce, and earned a grudging admiration from the Hungarians for its defiance of the World War I allies. In 1940 Hungary allowed the German army to cross its territory and, as a Yugoslavia. Hungary was the first country to draw back its West. In 1997 Hungary was NATO."
\caption{A summary with pruning that was too aggressive.}\label{badsummary}
\end{figure}

\begin{figure}
	
``Around the third century b.c., Hungary was occupied by a tribe known as Eraviscans, refugees from wars in Greece. Hungary remained beyond the reach of western Europe until the first century a.d., when the Roman legions advanced and pushed its northeast frontier to the Danube. Hungary was effectively dismembered: the north and west fell to the Transylvania became a so-called independent principality under Turkish auspices; and central Hungary came under direct Turkish rule. In 1867, under a compromise designed to curtail home-rule agitation, the empire was established and Hungary was finally granted its own government, though key ministries were shared with the Austrians. Meanwhile, the aftermath of war, as dictated by the Treaty of Trianon, was to cost Hungary very dearly. Germany, meanwhile, provided investment in Hungarian industry and a market for Hungarian farm produce, and earned a grudging admiration from the Hungarians for its defiance of the World War I allies. In 1940 Hungary allowed the German army to cross its and, as a reward, they temporarily recovered parts of its former lands from Romania and Yugoslavia. Hungary was the first country to draw back its Iron Curtain, dismantling the barbed wire along its border and allowing East Germans to escape to the West. In 1997 Hungary was offered membership in NATO."
\caption{The final summary for a document on the history of Budapest.}\label{goodsummary}
\end{figure}


 


	Given that document summarization is an unsupervised learning problem, it was particularly difficult to evaluate the ``success" of our program. However, we did have 2 different methods for evaluation in mind. First we have ROUGE, a program built by a USC grad student which provides a measure of how ``good" a summary is. We acquired this program and installed in on my laptop, but upon inspection of Chin-Yew Lin (the creator of ROUGE)'s paper \emph{ROUGE: A Package for Automatic Evaluation of Summaries}, we realized that ROUGE was made for the evaluation of abstract rather than extract summaries, and under the hood it operated using an $n$-gram model. Since we use sentences copied (albeit somewhat edited) from the original document, this method would be confounding and produce poor evaluation results.
	
We also considered a process by which one of us writes questions about a document, while the other one writes a summary. We would then randomly assign some human subjects either a human-made summary or a summary assembled by our program, and pose the questions that were written about that document. We would look at the the accuracy of their responses for both the human- and computer-generated summaries and compare the two. However, given the task of assembling a meaningful number of human subjects on which to run this test is outside of the scope of this project.

Given how difficult it is to evaluate how good a summary is, perhaps the best way to evaluate a summary is the eye test. As such, we've included three versions of the same summary with different pruning. Figure~\ref{rawsummary} gives the raw extract summary with no pruning. We can see that it does a pretty good job at grabbing sentences that are relevant to the document, though it could clearly use some pruning. Figure~\ref{badsummary} shows the first attempt at pruning, which computes the centrality of each subtree and takes the best for each sentence. As we can see, this is good at picking out key words, but destroys the meaning of the sentence. Figure~\ref{goodsummary}, our final implementation, strikes a good balance between the two by weighting each subtree by the log of its length. Examining the first sentence of each gives a good feel for how each does. The first sentence of Figure~\ref{rawsummary} has a lot of information but is overly specific for a summary. The first sentence of Figure~\ref{badsummary} is just hilariously wrong. The first sentence of Figure~\ref{goodsummary} prunes some of the unnecessary information from the sentence, but leaves the overall structure intact.



\section{Conclusion}
	Our initial goal in this project was to successfully implement single-document text summarization. However, this problem and its solutions proved to be so well-documented that simply implementing basic algorithms that already existed was not sufficiently interesting for our vision of the project. As a result, we implemented a nontrivial implementation of content selection, experimented with a potential improvement to sentence ordering, and worked out our own implementation of sentence realization. We were largely successful in our goals; while our experiment did not improve sentence ordering, we implemented a nontrivial content selection algorithm, and we put together and implemented a highly successful sentence realization algorithm without much external guidance. Ultimately, our project was successful in both producing a working single-document extract summarizer, and in experimenting with and assessing our own original approaches and modifications.


\end{document}



